\documentclass{beamer}

\usepackage[english]{babel}
\usepackage[UTF8]{ctex}              %lualatex用ctex
% \usepackage[utf8]{inputenc} 

%插入代码
\usepackage{listings}
\usepackage{fontspec}
\setmonofont{Monaco}
\usepackage{xcolor}
% 定义可能使用到的颜色
\definecolor{CPPLight}  {HTML} {686868}
\definecolor{CPPSteel}  {HTML} {888888}
\definecolor{CPPDark}   {HTML} {262626}
\definecolor{CPPBlue}   {HTML} {4172A3}
\definecolor{CPPGreen}  {HTML} {487818}
\definecolor{CPPBrown}  {HTML} {A07040}
\definecolor{CPPRed}    {HTML} {AD4D3A}
\definecolor{CPPViolet} {HTML} {7040A0}
\definecolor{CPPGray}  {HTML} {B8B8B8}
\lstset{
    columns=fixed,       
    frame=none,                                          % 不显示背景边框
    backgroundcolor=\color[RGB]{245,245,244},            % 设定背景颜色
    keywordstyle=\color[RGB]{40,40,255},                 % 设定关键字颜色
    numberstyle=\footnotesize\color{darkgray},           % 设定行号格式
    commentstyle=\it\color[RGB]{0,96,96},                % 设置代码注释的格式
    stringstyle=\rmfamily\slshape\color[RGB]{128,0,0},   % 设置字符串格式
    showstringspaces=false,                              % 不显示字符串中的空格
    language=c++,                                        % 设置语言
    morekeywords={alignas,continute,friend,register,true,alignof,decltype,goto,
    reinterpret_cast,try,asm,defult,if,return,typedef,auto,delete,inline,short,
    typeid,bool,do,int,signed,typename,break,double,long,sizeof,union,case,
    dynamic_cast,mutable,static,unsigned,catch,else,namespace,static_assert,using,
    char,enum,new,static_cast,virtual,char16_t,char32_t,explict,noexcept,struct,
    void,export,nullptr,switch,volatile,class,extern,operator,template,wchar_t,
    const,false,private,this,while,constexpr,float,protected,thread_local,
    const_cast,for,public,throw,std},
    emph={map,set,multimap,multiset,unordered_map,unordered_set,
    unordered_multiset,unordered_multimap,vector,string,list,deque,
    array,stack,forwared_list,iostream,memory,shared_ptr,unique_ptr,
    random,bitset,ostream,istream,cout,cin,endl,move,default_random_engine,
    uniform_int_distribution,iterator,algorithm,functional,bing,numeric,},
    emphstyle=\color{CPPViolet},
    basicstyle=\footnotesize\ttfamily,
}



\usepackage{graphicx} %插入图片的宏包
\usepackage{float} %设置图片浮动位置的宏包
\usepackage{subfigure} %插入多图时用子图显示的宏包



%Theme
% customize your own color and navigation
\usetheme[RGB={12 72 66}]{Simple} % color
\useoutertheme{miniframes}              % navigation
% \usetheme{Berlin}
% \usecolortheme{beaver}
% \input{pkgs.tex}



\author{syh.hs}
\title{LUT-ICPC Day 1 语法\&基础算法-1}
\date{\today}
\institute{兰州理工大学}

\AtBeginSection[]
{
  \begin{frame}
    \frametitle{Contents}
    \tableofcontents[currentsection]
  \end{frame}
}

\begin{document}
  % \maketitle

  % \tableofcontents
  

  % title
  % \frame[plain]{\titlepage}
  \begin{frame}[plain]
    \maketitle
    \begin{figure}[htbp] %H为当前位置,!htb为忽略美学标准,htbp为浮动图形
    \centering %图片居中
    \includegraphics[width=0.3\textheight,height=0.2\textwidth]{graph/new- 2.jpg} %插入图片,[]中设置图片大小,{}中是图片文件名
    % \caption{双叶天下第一可爱} %最终文档中希望显示的图片标题
    % \label{Fig.main2} %用于文内引用的标签
    \end{figure}
  \end{frame}

  % \begin{frame}
  %   \frametitle{Content}
  %   \tableofcontents % 这个命令将会显示目录
  % \end{frame}
  
  \section{前置}

  \begin{frame}
    \frametitle{知识准备}
    基础的递归、模拟

    基本时间复杂度会看,有概念

    会写暴力

    视情况而定,哪里需要补充哪里的知识,临时讲一下即可
  \end{frame}
  \begin{frame}
    \frametitle{训练}
    \href{https://en.cppreference.com/w/}{C++ Reference}

    \href{https://oi-wiki.org/}{OI Wiki}

    知乎

    每场CF的题解 \& 评论区,大佬的过题代码

    \vspace*{1\baselineskip}
    
    上述有不懂的进一步百度 | Google、交流,主要是坚持做题,少做水题、一眼题
    
    \vspace*{1\baselineskip}
    
    各种比赛省赛级别获奖对比赛经验和知识准备要求不高,当然之后如果甘肃有ICPC的省赛,和兰大同台竞技另说、、
    
    \vspace*{1\baselineskip}

    像ZJ等省份,XCPC有省赛,拿金的难度可能还要大于区域银
  \end{frame}

  \begin{frame}
    \frametitle{题单}
    \href{https://www.luogu.com.cn/training/list?type=official&page=1}{洛谷官方题单}

    \href{https://www.luogu.com.cn/training/2018}{2020,2021CF简单题精选} 适合有基础之后提高

    \vspace*{1\baselineskip}
    
    \href{https://www.luogu.com.cn/training/9376}{搜索题单}
    
    \vspace*{1\baselineskip}
    
    关于做题和看题解.. 不会做看题解是正常的,尽量是一点点看,理解思路,然后尝试自己写出,不行再看代码,理解了之后再写,注意抄题解和看题解的区分,没有太多思考内容的题一般10-15min
  \end{frame}

  \section{cpp98 \texorpdfstring{$\to$}\ cpp11}

  \begin{frame}
    \frametitle{summary}
    特定函数的部分主要看的是这篇
    \href{https://www.luogu.com.cn/blog/AccRobin/grammar-candies}{洛谷日报}
  
    \vspace*{2\baselineskip}
  
    模板,别的各种内容来自平时整理
  
  \end{frame}
  
  \begin{frame}
    \frametitle{algorithm库}
    常见的函数有swap,sort,unique,reverse,lower\_bound,upper\_bound等...
    
    一些别的函数

    std::find

    std::fill 一般用来弥补memset不能赋值的问题

    std::max\_element | min\_element (bg, ed) 第三个参数可传入比较函数

    std::count(bg,ed,val)
  
    std::count\_if(bg,ed,func) 常用func有isdigit,islower,isupper等

    std::for\_each(bg, ed, func)
  \end{frame}

  \begin{frame}
    \frametitle{numeric库}
    \begin{block}{std::accumulate(bg, ed, val)}
      可以用于序列求和,注意传参时val的类型避免溢出,第四个参数可以作为加法
    \end{block}

    \begin{block}{std::partial\_sum(bg1, ed1, bg2)}
      用于求前缀和,可以传入第四个参数作加法
    \end{block}

    \begin{block}{std::adjacent\_difference(bg1, ed1, bg2)}
      用于求差分,可以传入第四个参数作减法
    \end{block}
    
  \end{frame}

  \begin{frame}
    \frametitle{cmath库}
      \begin{block}{exp(x)}
        返回 $e^x$,$x$的有效范围是 $[−708.4,709.8]$
      \end{block}

      \begin{block}{log(x)}
        返回 $\ln{x}$,在 $x\leq0$时候报错,别的还有$log10$和$log2$,其中$log2$为C++11开始才有
      \end{block}

      floor 取上整

      ceil 取下整
  \end{frame}

  \begin{frame}
    \frametitle{GNU}

    这些内容不在C++标准中,如clang等其他编译器里可能没有,一般比赛提供的编译器都是GUN C++,也就是你们dev里面自带的mingw(Minimalist GNU for Windows)
    
  \end{frame}

  \begin{frame}
    \frametitle{GNU}

    \begin{block}{\_\_builtin 函数}

      \_\_builtin\_popcount(x) 统计二进制下x中1的个数

      \vspace*{1\baselineskip}
      
      \_\_builtin\_parity(x) 统计二进制下x中1个数的奇偶性
      
      \vspace*{1\baselineskip}
      
      \_\_builtin\_ffs(x) 统计二进制下最后一个1是从左往右第几位
      
      \vspace*{1\baselineskip}
      
      \_\_builtin\_ctz(x) 返回二进制下后导0的个数
      
      \vspace*{1\baselineskip}
      
      \_\_builtin\_clz(x) 返回二进制下前导0的个数
    \end{block}
    
  \end{frame}

  \begin{frame}
    \frametitle{..}
      大多函数其实也没啥用,主要是压行,节省时间

      Codeforces打多了就能感受到
  \end{frame}

  \begin{frame}
    \frametitle{lambda表达式}
    \begin{block}{example}
      作用是提供一个匿名函数
      
      sort(a.begin(), a.end(), [\&](type a, type b) {return ..});

      这里与手写一个cmp填入作用相同
    \end{block}

    \begin{block}{definition}
      [ captures ] ( params ) specifiers(可选) -> ret(可选) { body }
      captures:捕获外部变量列表。

      params:形参列表。
      
      specifiers:指定符序列。C++11 中常用的为 mutable(允许 body 修改以值捕获的参数)。
      
      ret:返回类型。若省略则由函数的 return 语句所隐含(或如函数不返回任何值则为 void)。
      
      body:函数体。
    \end{block}
  \end{frame}

  \begin{frame}[fragile]
    \frametitle{lambda表达式}
    竞赛里一般不考虑specifier

    \begin{block}{example}
      \begin{lstlisting}
int factorA = 1, factorB = 3;
std::vector<int> c(10);
std::iota(c.begin(), c.end(), 0);
for (auto it = c.begin(); it != c.end(); ++it)
  std::cout << *it << " \n"[it == std::prev(c.end())];
for_each(c.begin(), c.end(), [&](int &x) {
  x += factorA & factorB;
});
for (auto it : c)
  std::cout << it << ' ';
        
      \end{lstlisting}

    \end{block}

    提一嘴vector<bool>,这个东西不建议使用,具体原因知乎搜索
  \end{frame}

  \begin{frame}[fragile]
    \frametitle{lambda表达式}
    另一种方式是关于递归,如何调用自己的问题
    \begin{block}{example}
      \begin{lstlisting}
auto sum = [](auto self, int x) -> int {
  return x == 1 ? 1 : x + self(self, x - 1);
};

std::cout << sum(sum, 10);
      \end{lstlisting}

    \end{block}

    一种方法是用固定的auto格式
    
    不过记得标明类型,否则可能自动推导失败

    \vspace*{1\baselineskip}

    而且从c++14开始才支持lambda中参数类型有auto,
    所以如果你要是c++11这么写递归就完了,比方说某绿桥杯、、
  \end{frame}


  \begin{frame}[fragile]
    \frametitle{lambda表达式}

    因为上述问题,所以我们想要在C++11中使用lambda的递归,需要用std::function封装该匿名函数

    \begin{block}{example}
      \begin{lstlisting}
std::function<int(int)> f = [&](int x) -> int {
  return x == 1 ? 1 : x + f(x - 1);
};
std::cout << f(3);
      \end{lstlisting}      
    \end{block}
    \pause
  
    更多内容移步 \href{https://oi-wiki.org/lang/lambda}{OI-wiki lambda表达式}
  \end{frame}


  \begin{frame}[fragile]
    \frametitle{随机数 \& shuffle}
    \begin{block}{definition}
  
      mt19937是一个随机数生成器类

      shuffle 需要提供头尾迭代器和一个随机数生成器作为参数

      随机数生成器需要一个种子来产生一个随机数,种子不同,随机结果才会不同

      一般用1970年(Unix纪元)至今秒数作为种子
      
      \begin{lstlisting}  
std::vector<int> a(10);
std::iota(a.begin(), a.end(), 0);
std::mt19937 myrand(time(0));
std::shuffle(a.begin(), a.end(), myrand);
for (auto it : a)
  std::cout << it << ' ';
      \end{lstlisting}
    \end{block}
  \end{frame}

  \begin{frame}
    \frametitle{模板}
      \begin{block}{关于template}
        知道template<typename T> $\mid$ template<class T>
        这两种比较基本的形式就可以,更多的知识和一些魔法技巧自行了解
      \end{block}
  \end{frame}

  \section{cpp14 \texorpdfstring{$\to$}\ cpp17}

  \begin{frame}[fragile]
    \frametitle{零碎的语法糖}
    
    C++14中 auto可以作为普通函数的返回值

    \vspace*{1\baselineskip}
    
    C++17中 支持结构化绑定
    
    搭配 range-for 使用:
    \begin{block}{e.g.}
      \begin{lstlisting}
vector<pair<int, int>> v(n);
for (auto&[x,y] : v) cin>>x>>y;
      \end{lstlisting}
    \end{block}
    
    用于带权图的遍历

  \end{frame}
  \begin{frame}[fragile]
    \frametitle{零碎的语法糖}

    C++17中支持省略类型的vector初始化方式

    \begin{block}{e.g.}
      \begin{lstlisting}
int n, m;
std::cin >> n >> m;
std::vector a(n, std::vector<int>(m, 1));
      \end{lstlisting}
    \end{block}
  \end{frame}

  \begin{frame}
    如果有想了解C++17的可以看

    C++17 in Detail
  \end{frame}

  \section{基础数据结构}

  \begin{frame}
    \frametitle{扯淡}

    语言特性的权重在竞赛学习中很低,只能起到锦上添花的作用,关键还是把题目做出来

    \vspace*{1\baselineskip}

    还有就是关于贪心、DP这一类思维题,最好直接Codeforces找greedy dp的标签,按照自身rating+200分的题开始做,别在概括性的套话和水题上纠结

  \end{frame}

  \begin{frame}
    \frametitle{扯淡}

    模拟题、字符串题怎么训练?

    \vspace*{1\baselineskip}

    打开Codeforces,点开Problem Set,加上对应tag,找合适分段的题目做
  \end{frame}

  \begin{frame}[fragile]
    \frametitle{前缀和\&差分}

    \begin{block}{前缀和}
      pre[i] = pre[i-1] + a[i]
      
      或者

      pre[i] = pre[i-1] xor a[i]
    \end{block}

    \begin{block}{二维前缀和}
      pre[i][j] = pre[i-1][j]+pre[i][j-1]-pre[i-1][j-1]+a[i][j]
    \end{block}

    \begin{block}{差分}
      minus[i] = a[i] - a[i-1]
    \end{block}
  \end{frame}

  \begin{frame}
    \frametitle{前缀和\&差分}
    还有树上差分+倍增
    
    \vspace*{1\baselineskip}

    树状数组 + 差分等等
  \end{frame}

  \begin{frame}

    开始单调线性结构,别的内容还有很多,有经验了之后自己去看wiki看文档吧..
    
    \vspace*{1\baselineskip}

    \frametitle{单调栈}

    (看情况给5-10min思考..)

    其实比较局限,只能解决一类很固定的问题

    因此也很容易看出来233

    \vspace*{1\baselineskip}
    
    \href{https://www.luogu.com.cn/problem/P5788}{模板题}

    \vspace*{1\baselineskip}

    \href{http://syh521.cn/file/monotone-stack.cpp}{对应代码}
  \end{frame}

  \begin{frame}


    \frametitle{单调栈}
    看一道题,\href{https://www.luogu.com.cn/problem/SP1805}{最大子矩形面积}

    \pause

    \begin{block}{题意}
      给定长为n的数列a,令$n\leq 10^5,a_i\leq 10^9$
  
      $\mathtt{sum=(r-l+1)*Min,Min=min\{a_i\mid i\in [l,r]\}}$,要求$\mathtt{sum}$的最大值

    \end{block}
    \pause
    问题的实质是啥
  \end{frame}

  \begin{frame}
    \frametitle{单调栈}
    
    对于每个i,我们希望找到对应的[l,r],保证a[l],a[l+1],$\cdots$,a[r]都大于等于a[i]

    \pause

    \vspace*{1\baselineskip}
    
    不难意识到两个单调栈就能解决该问题
    
    \pause
    
    \vspace*{1\baselineskip}

    自己动手试试!
  \end{frame}

  \begin{frame}
    \frametitle{单调队列}

    \href{https://www.luogu.com.cn/problem/P1886}{模板题}

    \begin{block}{题意}
      给定长为$n$的序列$a$,以及一个$k$,窗口从$[1,k]$一直滑动到$[n-k+1,n]$

      我们想知道每次窗口内的最大 / 最小值
    \end{block}
  \end{frame}

  \begin{frame}[fragile]
    \frametitle{单调队列}
    以求解最大值为例

    可以发现,当前区间内靠后的,偏大的数字,比靠前的且偏小的数字要好,我们可以不保存后者

    \pause

    \vspace*{1\baselineskip}

    如果要保存靠前的数字,那么一定是他比后面的数字大

    \pause

    那么像是搞一个单调递减,下标递增的序列,类似
    \begin{lstlisting}
deque<pair<int,int>>
    \end{lstlisting}
    ,来维护区间最大值

    \pause

    对于新来的一个数$x$
    ,我们要不停的扔掉队列末端比$x$小的数,因为那些已经没有了保存的价值
    ,这样就维护了一个单调性

    \vspace*{1\baselineskip}

    同时检索头部的元素是否还在区间内,不在就pop掉
  \end{frame}

  \begin{frame}
    “如果一个选手比你小还比你强,你就可以退役了。”
    \rightline{——单调队列原理[doge]}
    
    \vspace*{1\baselineskip}
    
    \href{http://syh521.cn/file/monotone-queue.cpp}{代码}
    
    \vspace*{1\baselineskip}

    单调队列可以维护左右端点都向一个方向移动时的区间最值,
    这个性质常用来做动态规划的优化
  \end{frame}

  \begin{frame}
    \frametitle{倍增 \& ST表}
    \href{https://www.luogu.com.cn/problem/P3865}{ST表模板题}

    \vspace*{1\baselineskip}

    \begin{block}{题意}
      给定$n,m$和不会改变的数组$\mathtt{a\{n\}}$,以及$m$个询问

      每次询问给定一个l,r,要求出$\mathtt{a[l] - a[r]}$中的最大值,
      $n\leq 10^5$,$m\leq 2*10^6,a_i\in[0,10^9]$
    \end{block}
  \end{frame}

  \begin{frame}[fragile]
    \frametitle{倍增 \& ST表}

    如果我们知道了两个区间$\mathtt{[l,r]}$和$\mathtt{[L,R]}$的最大值,
    这里不妨设$l \leq L \leq r \leq R$
    ,那么就可以得到一个更大的区间,$[l,R]$中的最值

    \pause

    \vspace*{1\baselineskip}

    我们在有了预处理基础的条件上,这里就是a[i]=0
    ,然后传递关系的时候成倍增长,即
    \begin{align*}
      \mathtt{max\{a[l,l+2^i-1]\}=max(}
      &\mathtt{max\{a[l,l+2^{i-1}-1]\},} \\
      & \mathtt{max\{a[l+2^{i-1},l+2^i-1]\})}
    \end{align*}

    我们设$\mathtt{f[i][j]}$表示 $\mathtt{[l,l+2^j-1]}$中的最大值,
    显然有
    $\mathtt{f[i][j]=max\{f[i][j-1],f[i+2^{j-1}][j-1]\}}$
  \end{frame}

  \begin{frame}
    \frametitle{倍增 \& ST表}
    \begin{block}{如何查询}
      对于给定的[L,R],我们要怎么查询最值?
    \end{block}

    \pause

    预处理 $\mathtt{log_2}$ 数组,或者调函数,
    令 $\mathtt{d=log_2{(R-L+1)}}$

    询问$\mathtt{max(f[L][d],f[L+2^d][d])}$即可

    \vspace*{1\baselineskip}
    
    这种预处理信息时成倍增长的手段,就叫倍增

    \vspace*{1\baselineskip}

    \href{http://syh521.cn/file/st-table.cpp}{代码}

    注意C++中位运算的优先级比加减要低,不确定时加上括号
  \end{frame}


  \begin{frame}
    \frametitle{另一种解法-ST表}
    回过去看 最大子矩形面积 这个问题,
    找到最小值作为决定的边界,然后递归两边

    \vspace*{1\baselineskip}

    每个数都会作为区间最小值被计算一次,总复杂度就是ST表的复杂度,
    预处理$\mathtt{O(NlogN)}$,查询$\mathtt{O(1)}$

    \vspace*{1\baselineskip}

    \href{http://syh521.cn/file/another-1.cpp}{代码}
    ,感兴趣的同学可以自己去SPOJ写一下交一下

    \vspace*{1\baselineskip}

    这也是个很好的例子,如何把各种STL运用起来


  \end{frame}

  \begin{frame}[fragile]
    \frametitle{另一种解法}
    还是考虑对于每个$\mathtt{a_i}$求出对应的左端点$\mathtt{L}$
    和右端点$\mathtt{R}$

    \pause

    \vspace*{1\baselineskip}
    
    先来处理一个点的左端点,发现似乎存在某种传递关系?
    
    \pause

    \vspace*{1\baselineskip}
    一开始每个点的最左端点都是自身,这毫无疑问,并且如果左端点左边的点$\mathtt{left(i)-1=L}$
    (如果存在的话)有$\mathtt{a[L]>a_i}$,那么可以把左端点扩展到 $\mathtt{left(L)}$ ,即

    \begin{lstlisting}
left[i]=i;
while (left[i]-1 >= 1 && a[left[i]-1]>=a[i])
  left[i]=left[left[i]-1]
    \end{lstlisting}

    \pause

    复杂度能过吗?

    \pause

    其实就是合并区间,一共只有n段区间可以合并,显然 $\mathtt{O(N)}$
  \end{frame}

  \begin{frame}
    \frametitle{另一种解法}
    笛卡尔树也可以解决本题

    感兴趣的同学上OI-wiki自取,内容并不难理解
  \end{frame}

  \begin{frame}
    \frametitle{堆}
      根据时间和观众意愿决定讲不讲

      用OI-wiki和写过的博客凑合一下,这里不写内容了..
  \end{frame}

  \begin{frame}
    \frametitle{并查集}
      同上
  \end{frame}

  \section{位运算|预处理瞎扯}
  \begin{frame}
    \frametitle{随便扯点}
    如果有时间的话,看情况,防止内容不够
  \end{frame}

  \begin{frame}
    \frametitle{预处理}

    顾名思义,提前处理信息然后重复使用,前缀和,差分都是此种思想的基本手段

  \end{frame}

  \begin{frame}
    \frametitle{位运算}

    首先是一个 int 有32位,去掉最高位是符号位,有31位可以表示数值

    那么每一位都是0,1,我们可以利用二进制的性质去做一些状态表示

    \begin{block}{example|0-index}
      除非你能记住优先级并且保证不出bug,不然建议加括号

      移位 $\mathtt{<<\ >>}$

      取出x的第i位 $\mathtt{(x>>i)\&1}$

      消除x的最后一个1 $\mathtt{x\&{(x-1)}}$

      交换两个数字(用于理解) $\mathtt{a\^{} =b\^{} =a\^{} =b}$

    \end{block}

    等等.. 位运算有很多技巧,\href{https://www.zhihu.com/question/38206659}{知乎}
    上可以找到很多资料,别看营销号就行了
  \end{frame}

  \begin{frame}
    \frametitle{预处理}
    直接对着题目干

    首先是两道昨晚的 Atcoder

    \href{https://atcoder.jp/contests/abc261/tasks/abc261_d}{Atcoder 261D}
    
    \href{https://atcoder.jp/contests/abc261/tasks/abc261_e}{Atcoder 261E}

    然后是最近的CF div2.B

    \href{https://codeforces.com/contest/1709/problem/B}{Educational Codeforces Round 132B}
  \end{frame}


  \section{作业}
  \begin{frame}
    \frametitle{写题}
    Codeforces Div.2 / Div. 3泛做,无难度要求,自己看着搞

    \vspace*{1\baselineskip}
    
    洛谷或其他平台上相关题目的练习

    \vspace*{1\baselineskip}

    晚上抽出时间打CF,如果早睡早起的可以virtual participation
  \end{frame}

  \begin{frame}[fragile]
    \frametitle{格式要求}

    写解题报告,每道题要有题目链接|来源、解题思路(随便写两句说的对就行)、代码

    \vspace*{1\baselineskip}
    
    可以是word,\LaTeX/Markdown生成pdf,博客,acwing打卡等任意形式

    \vspace*{1\baselineskip}

    没有数量要求,一周交,自愿为主,只做题不写也可
  \end{frame}

\end{document}