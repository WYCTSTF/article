\documentclass{beamer}

\usepackage[english]{babel}
\usepackage[UTF8]{ctex}              %lualatex用ctex
% \usepackage[utf8]{inputenc} 

%插入代码
\usepackage{listings}
\usepackage{fontspec}
\setmonofont{Monaco}
\usepackage{xcolor}
% 定义可能使用到的颜色
\definecolor{CPPLight}  {HTML} {686868}
\definecolor{CPPSteel}  {HTML} {888888}
\definecolor{CPPDark}   {HTML} {262626}
\definecolor{CPPBlue}   {HTML} {4172A3}
\definecolor{CPPGreen}  {HTML} {487818}
\definecolor{CPPBrown}  {HTML} {A07040}
\definecolor{CPPRed}    {HTML} {AD4D3A}
\definecolor{CPPViolet} {HTML} {7040A0}
\definecolor{CPPGray}  {HTML} {B8B8B8}
\lstset{
    columns=fixed,       
    frame=none,                                          % 不显示背景边框
    backgroundcolor=\color[RGB]{245,245,244},            % 设定背景颜色
    keywordstyle=\color[RGB]{40,40,255},                 % 设定关键字颜色
    numberstyle=\footnotesize\color{darkgray},           % 设定行号格式
    commentstyle=\it\color[RGB]{0,96,96},                % 设置代码注释的格式
    stringstyle=\rmfamily\slshape\color[RGB]{128,0,0},   % 设置字符串格式
    showstringspaces=false,                              % 不显示字符串中的空格
    language=c++,                                        % 设置语言
    morekeywords={alignas,continute,friend,register,true,alignof,decltype,goto,
    reinterpret_cast,try,asm,defult,if,return,typedef,auto,delete,inline,short,
    typeid,bool,do,int,signed,typename,break,double,long,sizeof,union,case,
    dynamic_cast,mutable,static,unsigned,catch,else,namespace,static_assert,using,
    char,enum,new,static_cast,virtual,char16_t,char32_t,explict,noexcept,struct,
    void,export,nullptr,switch,volatile,class,extern,operator,template,wchar_t,
    const,false,private,this,while,constexpr,float,protected,thread_local,
    const_cast,for,public,throw,std},
    emph={map,set,multimap,multiset,unordered_map,unordered_set,
    unordered_multiset,unordered_multimap,vector,string,list,deque,
    array,stack,forwared_list,iostream,memory,shared_ptr,unique_ptr,
    random,bitset,ostream,istream,cout,cin,endl,move,default_random_engine,
    uniform_int_distribution,iterator,algorithm,functional,bing,numeric,},
    emphstyle=\color{CPPViolet},
    basicstyle=\footnotesize\ttfamily,
}



\usepackage{graphicx} %插入图片的宏包
\usepackage{float} %设置图片浮动位置的宏包
\usepackage{subfigure} %插入多图时用子图显示的宏包
% \usepackage[colorlinks,linkcolor=blue]{hyperref}


%Theme
% customize your own color and navigation
\usetheme[RGB={12 72 66}]{Simple} % color
\useoutertheme{miniframes}              % navigation
% \usetheme{Berlin}
% \usecolortheme{beaver}
% \input{pkgs.tex}



\author{沈宇昊}
\title{LUT 2023ICPC校赛题解}
\date{\today}
\institute{兰州理工大学}

\AtBeginSection[]
{
  \begin{frame}
    \frametitle{Contents}
    \tableofcontents[currentsection]
  \end{frame}
}

\begin{document}
  % \maketitle

  % \tableofcontents
  

  % title
  % \frame[plain]{\titlepage}
  \begin{frame}[plain]
    \maketitle
    % \begin{figure}[htbp] %H为当前位置,!htb为忽略美学标准,htbp为浮动图形
    % \centering %图片居中
    % \includegraphics[width=0.3\textheight,height=0.2\textwidth]{graph/shuangye_.png} %插入图片,[]中设置图片大小,{}中是图片文件名
    % % \caption{双叶天下第一可爱} %最终文档中希望显示的图片标题
    % % \label{Fig.main2} %用于文内引用的标签
    % \end{figure}
  \end{frame}

  \begin{frame}
    \frametitle{Content}
    \tableofcontents % 这个命令将会显示目录
  \end{frame}

  \section{begin}
  \begin{frame}
    如果你已经写过ICPC区域赛的题,去看过解题报告,或者学某些知识看别人的博客时,往往会碰到别人三言两语讲完一个思路,自己还是一头雾水的情况。

    \vspace*{1\baselineskip}
    
    但竞赛的状态就是如此,大部分时候并没有十分精巧且详略得当的最优攻略。
    
    \vspace*{1\baselineskip}

    但无需因此止步,或寻求事无巨细的教程而踟蹰不前,或浮夸潦草,人云亦云,复制题解,当个报菜名的大师。二者都是不可取的。
  \end{frame}

  \begin{frame}
    这份题解也是同样的道理,只会告诉你解题的方法,但是不会将方法教给你。等待教学固然是一个途径,但是自己有了需求再去努力会更得心应手
  \end{frame}

  \section{A - 签到题}
  \begin{frame}
    编译期
    $\textmd{a[i]}$的形式会被翻译成$\textmd{*(a+i)}$,展开后不难发现是三次映射

    \vspace*{1\baselineskip}

    得到$\textmd{i[a][a][a]}$等价$\textmd{a[a[a[i]]]}$
  \end{frame}

  \section{B - 简单的连通问题}

  \begin{frame}
    如果你不知道图论的\href{https://oi-wiki.org/graph/concept/}{基本概念}

    \vspace*{1\baselineskip}
    
    如果你不会\href{https://oi-wiki.org/graph/save/}{存图}

    \vspace*{1\baselineskip}
    
    原题地址是 \href{https://atcoder.jp/contests/abc292/tasks/abc292_d}{atcoder292d},可以去做一下

    \vspace*{1\baselineskip}

    题面出了重大问题,是要考虑重边和自环,而不是不要考虑,原来的明显和样例冲突了
  \end{frame}

  \begin{frame}
    正常dfs一遍,把各连通块的点数和边数数一下即可

    \vspace*{1\baselineskip}

    \href{https://syh521.cn/file/at292d.cc}{代码},仅作参考,对于新手而言还是值得自己写一写的
  \end{frame}

  \section{C - 小叶的约会}
  \begin{frame}
  
    动态维护两个有序的数列,很容易想到set和二分。

    剩下的,看代码吧....

    \vspace*{1\baselineskip}
    
    \href{https://syh521.cn/file/u297129.cc}{代码}
  \end{frame}

  \section{D - 翘课计划}
  \begin{frame}
    树状数组、线段树,甚至因为单点修改的缘故ST表。。一堆都能写。

    \vspace*{1\baselineskip}

    但是有这个必要吗?

    \vspace*{1\baselineskip}
    
    考虑一前一后两个数,对于任何大小的$L$,如果后面的数字更大,那么无论什么情况前面的数都不可能成为答案。
    
    \vspace*{1\baselineskip}
    
    自然的,考虑维护一个答案候选序列,从前往后值依次减小,下标依次增大(单调队列)。
    
    \vspace*{1\baselineskip}

    查询的时候因为单调性,二分一个位置即可,不会单调队列的可以看群里之前的课件和录屏
  \end{frame}

  \begin{frame}
    单调队列+二分答案,$O(M+M\log{M})$,\href{https://syh521.cn/file/d-bound.cpp}{代码}

    \vspace*{1\baselineskip}

    是否有更简单的做法?

    \pause

    \vspace*{1\baselineskip}
    
    考虑并查集,维护单调队列的时候,被更新的值的父亲指向更新他们的值。以此类推,就起到了一个传递的作用。复杂度$\textmd{O(M)}$
    
    \vspace*{1\baselineskip}

    自己动手试试!
  \end{frame}

  \section{E - 简单的异或题}

  \begin{frame}
    01trie模板题,请自行在OI-wiki学习trie树

    \vspace*{1\baselineskip}

    trie的节点表示状态,而边则是转移的方式。考虑数的二进制,从高到低考虑位数,我们很自然的维护了一个已有的数字的trie,存在各个节点中。

    \vspace*{1\baselineskip}
    
    根据异或同0异1的原则,我们拿到新的数字$x$,从高到低的在trie中找每一位和$x$不同的数字,如果找不到则找相同的情况,这必然存在,因为二进制上不是0就是1,考虑32位,肯定会被保存下来。

    \vspace*{1\baselineskip}

    同时因为高位最优,自然满足一个贪心的原则

    \vspace*{1\baselineskip}

    \href{https://syh521.cn/file/e.cpp}{代码}
  \end{frame}

  \section{F - 简单的数学题}

  \begin{frame}
    问朋友要的题,一开始把式子看颠倒了。事实上和整除分块没有什么关系,如果交换分子分母好像是个数论分块+前缀和...

    \vspace*{1\baselineskip}

    枚举$\sum$里的东西,然后确定k的上下界算个贡献即可.
  \end{frame}

  \begin{frame}[fragile]
\begin{lstlisting}
long long get(int n) {
long long res = 0;
  for (int m = 1; 114 * m * m <= n; m++) {
    int l = 114 * m * m,
        r = min(114 * (m + 1) * (m + 1) - 1, n);
    res += (r - l + 1) * m;
  }
  return res;
}
\end{lstlisting}

  \pause

  \vspace*{1\baselineskip}

  求一个$\mathtt{get(n+514)-get(n)}$就好了
  \end{frame}

  \section{G - 牛牛的括号}

  \begin{frame}
    原题 CSP-S 2019 Day1 T2 括号树

    \vspace*{1\baselineskip}
    
    自己找题解看吧.. 写不了就放了
    
    \vspace*{1\baselineskip}

    关键在于对一道题的尝试,以此题为例,如果一开始没有任何的思路,那么先考虑纯链的情况,然后是尝试自己写几个样例,试着找找树上的规律。尽可能的去找方向思考,而不是一时半会儿想不到正解就直接去看答案,这样对训练也没有什么效果可言
  \end{frame}

  \section{end}

  
  \begin{frame}
    \frametitle{总结}
  
      如果你不会STL,对各种数据结构不熟悉,也不了解C++的各种写法,递归写的也不熟练,甚至不会,这都不是什么大问题。或者看书,或者按洛谷、acwing等OJ的题单教程来熟悉,都是可以的
      
      \vspace*{1\baselineskip}

      基础的内容没什么好"学"的,主要靠写题来熟练。

      \vspace*{1\baselineskip}

      真正重要的是克服恐惧心理,勇敢尝试去理解一些陌生的,抽象的概念,发掘其中的乐趣,而不是找借口,或者回避,纵使不打竞赛,干别的事情,搞项目,学技术,也是同样的道理。
  
  \end{frame}

  \begin{frame}
    至于平时的训练,该不该看题解。一道题要想多久看题解,诸如此类的问题,自己写的时间久了自然会有答案,我个人的建议是思考15分钟左右。当然还是酌情,掌握了一定知识、套路的情况下可以多想一想。

    \vspace*{1\baselineskip}
    
    另外,训练的时候要避免眼高手低。切忌没有怎么训练就搞总结性发言,动辄不做模板题,想着要练所谓的“思维”,然而真要从空的cpp开始ac一个套路题,又不能成功,这不是一件好事。
    
    \vspace*{1\baselineskip}

    希望这场比赛对你有所帮助
  \end{frame}
\end{document}