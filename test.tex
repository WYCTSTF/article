\documentclass{beamer}

\usepackage[english]{babel}
\usepackage[UTF8]{ctex}              %lualatex用ctex
% \usepackage[utf8]{inputenc} 

%插入代码

\usepackage{listings}
\usepackage{xcolor}
% 定义可能使用到的颜色
\definecolor{CPPLight}  {HTML} {686868}
\definecolor{CPPSteel}  {HTML} {888888}
\definecolor{CPPDark}   {HTML} {262626}
\definecolor{CPPBlue}   {HTML} {4172A3}
\definecolor{CPPGreen}  {HTML} {487818}
\definecolor{CPPBrown}  {HTML} {A07040}
\definecolor{CPPRed}    {HTML} {AD4D3A}
\definecolor{CPPViolet} {HTML} {7040A0}
\definecolor{CPPGray}  {HTML} {B8B8B8}
\lstset{
    columns=fixed,       
    numbers=left,                                        % 在左侧显示行号
    frame=none,                                          % 不显示背景边框
    backgroundcolor=\color[RGB]{245,245,244},            % 设定背景颜色
    keywordstyle=\color[RGB]{40,40,255},                 % 设定关键字颜色
    numberstyle=\footnotesize\color{darkgray},           % 设定行号格式
    commentstyle=\it\color[RGB]{0,96,96},                % 设置代码注释的格式
    stringstyle=\rmfamily\slshape\color[RGB]{128,0,0},   % 设置字符串格式
    showstringspaces=false,                              % 不显示字符串中的空格
    language=c++,                                        % 设置语言
    morekeywords={alignas,continute,friend,register,true,alignof,decltype,goto,
    reinterpret_cast,try,asm,defult,if,return,typedef,auto,delete,inline,short,
    typeid,bool,do,int,signed,typename,break,double,long,sizeof,union,case,
    dynamic_cast,mutable,static,unsigned,catch,else,namespace,static_assert,using,
    char,enum,new,static_cast,virtual,char16_t,char32_t,explict,noexcept,struct,
    void,export,nullptr,switch,volatile,class,extern,operator,template,wchar_t,
    const,false,private,this,while,constexpr,float,protected,thread_local,
    const_cast,for,public,throw,std},
    emph={map,set,multimap,multiset,unordered_map,unordered_set,
    unordered_multiset,unordered_multimap,vector,string,list,deque,
    array,stack,forwared_list,iostream,memory,shared_ptr,unique_ptr,
    random,bitset,ostream,istream,cout,cin,endl,move,default_random_engine,
    uniform_int_distribution,iterator,algorithm,functional,bing,numeric,},
    emphstyle=\color{CPPViolet}, 
}

% customize your own color and navigation
\usetheme[RGB={12 72 66}]{Simple} % color
\useoutertheme{miniframes}              % navigation

% \usetheme{Berlin}
% \usecolortheme{beaver}
% \usepackage[english]{babel}
\usepackage{fancyhdr}        % header footer
\usepackage{graphicx}        % figure
\usepackage{algorithm2e}
\usepackage{booktabs}
\usepackage{xcolor}
\usepackage{bookmark}


\author{沈宇昊}
\title{LUT-XCPC Day 1 语法\&基础数据结构}
\date{\today}
\institute{兰州理工大学}

\AtBeginSection[]
{
  \begin{frame}
    \frametitle{Contents}
    \tableofcontents[currentsection]
  \end{frame}
}

\begin{document}

  % \maketitle

  % \tableofcontents
  
  \frame[plain]{\titlepage}
  
  % \begin{frame}
  %   \frametitle{Content}
  %   \tableofcontents % 这个命令将会显示目录
  % \end{frame}
  
  \section{前置}

  % \frame[plain]{\section{前置}}

  \begin{frame}
    \frametitle{知识准备}
    基础的递归、模拟

    基本时间复杂度会看,有概念

    会写暴力

    视情况而定,哪里需要补充哪里的知识,临时讲一下即可
  \end{frame}
  \begin{frame}
    \frametitle{资料 \& 训练方式}
    \href{https://en.cppreference.com/w/}{C++ Reference}

    \href{https://oi-wiki.org/}{OI Wiki}

    知乎

    每场CF的题解 \& 评论区,大佬的过题代码

    \vspace*{1\baselineskip}
    
    上述有不懂的进一步百度 | Google、交流,主要是坚持做题
    
    \vspace*{1\baselineskip}
    
    各种比赛省赛级别获奖对比赛经验和知识准备要求不高,当然之后如果甘肃有ICPC的省赛,和兰大同台竞技另说、、
    
    \vspace*{1\baselineskip}

    像ZJ等省份,XCPC有省赛,拿金的难度可能还要大于区域银
  \end{frame}

  \begin{frame}
    \frametitle{题单}
    \href{https://www.luogu.com.cn/training/list?type=official&page=1}{洛谷官方题单}

    \href{https://www.luogu.com.cn/training/2018}{2020,2021CF简单题精选} 适合有基础之后提高

    \vspace*{1\baselineskip}
    
    \href{https://www.luogu.com.cn/training/9376}{搜索题单}
    
    \vspace*{1\baselineskip}
    
    关于做题和看题解.. 不会做看题解是正常的,尽量是一点点看,理解思路,然后尝试自己写出,不行再看代码,理解了之后再写,注意抄题解和看题解的区分,没有太多思考内容的题一般10-15min
  \end{frame}

  \section{cpp98 \texorpdfstring{$\to$} cpp11}

  \begin{frame}  
    \frametitle{summary}
    特定函数的部分主要看的是这篇
    \href{https://www.luogu.com.cn/blog/AccRobin/grammar-candies}{洛谷日报}
  
    \vspace*{2\baselineskip}
  
    模板,别的各种内容来自平时整理
  
  \end{frame}
  
  \begin{frame}
    % \frametitle{'algorithm库'}
    \frametitle{algorithm库}
    常见的函数有swap,sort,unique,reverse,lower\_bound,upper\_bound等...
    
    一些别的函数

    std::find

    std::fill 一般用来弥补memset不能赋值的问题

    std::max\_element | min\_element (bg, ed) 第三个参数可传入比较函数

    std::count(bg,ed,val)
  
    std::count\_if(bg,ed,func) 常用func有isdigit,islower,isupper等

    std::for\_each(bg, ed, func)
  \end{frame}

  \begin{frame}
    \frametitle{numeric库}
    \begin{block}{std::accumulate(bg, ed, val)}
      可以用于序列求和,注意传参时val的类型避免溢出,第四个参数可以作为加法
    \end{block}

    \begin{block}{std::partial\_sum(bg1, ed1, bg2)}
      用于求前缀和,可以传入第四个参数作加法
    \end{block}

    \begin{block}{std::adjacent\_difference(bg1, ed1, bg2)}
      用于求差分,可以传入第四个参数作减法
    \end{block}
    
  \end{frame}

  \begin{frame}
    \frametitle{cmath库}
      \begin{block}{exp(x)}
        返回 $e^x$,$x$的有效范围是 $[−708.4,709.8]$
      \end{block}

      \begin{block}{log(x)}
        返回 $\ln{x}$,在 $x\leq0$时候报错,别的还有$log10$和$log2$,其中$log2$为C++11开始才有
      \end{block}

      floor 取上整

      ceil 取下整
  \end{frame}

  \begin{frame}
    \frametitle{GNU}

    这些内容不在C++标准中,如clang等其他编译器里可能没有,一般比赛提供的编译器都是GUN C++,也就是你们dev里面自带的mingw(Minimalist GNU for Windows)
    
  \end{frame}

  \begin{frame}
    \frametitle{GNU}

    \begin{block}{\_\_builtin 函数}

      \_\_builtin\_popcount(x) 统计二进制下x中1的个数

      \vspace*{1\baselineskip}
      
      \_\_builtin\_parity(x) 统计二进制下x中1个数的奇偶性
      
      \vspace*{1\baselineskip}
      
      \_\_builtin\_ffs(x) 统计二进制下最后一个1是从左往右第几位
      
      \vspace*{1\baselineskip}
      
      \_\_builtin\_ctz(x) 返回二进制下后导0的个数
      
      \vspace*{1\baselineskip}
      
      \_\_builtin\_clz(x) 返回二进制下前导0的个数
    \end{block}
    
  \end{frame}

  % \begin{frame}
  %   \frametitle{Test}
  %   QAQ\\
  %   \pause
  %   QwQ\\
  %   \pause
  %   QvQ
  % \end{frame}
\end{document}